\chapter{System Administration}

\section{Archives and Compressed Files}

In any operating system it is a common task to work with compressed and archive files.  In the Windows world these are usually in the .zip, .rar and .cab formats.  While these files can be created, extracted and worked with in Unix, they are not as common as some others, particularly for software distribution.  It is therefore important to know about these formats and how to work with them.  Using .zip, .rar and .7z files in Unix also usually requires the installation of third party software, but this is typically as simple as installing from a repository (see Installing Software).

\subsection{tar}

\textbf{tar} is a very common archive file format.  In and of itself, it does not perform any compression.  tar was originally and can still be used for creating backup tape archives (hence, the name: "t"ape "ar"chive).  But more commonly today tar is used to create simple archive files.  This allows one to create one file that contains multiple others (including other tar files).\\

To create a tar file, use a command with the following structure:

\begin{verbatim}
$ tar -cvf destination.tar  sourcefile1 sourcefile2 sourcefile3 ...
\end{verbatim}

This will create a new file called destination.tar that contains all of the source files listed after it.  The source files may also be directories.  For example:

\begin{verbatim}
$ tar -cvf music.tar Music/
\end{verbatim}

This will create a tar file called music.tar that contains the Music directory and all of its contents.  For those curious, the arguments mean the following: c - create, v - verbose (show information) and f - destination file.

As always, see the tar man page for complete documentation.

\subsection{gzip, bzip2}

Creating an archive is not always enough.  While Windows users are accustomed to normally using the ZIP file format, which compresses and archives, tar files are not compressed.  This allows the Unix user to choose the compression utility.  While there are many different compression algorithms and tools, by far the most common are gzip and bzip2.\\

gzip is probably a bit more common than bzip2.  While bzip2's compression is usually (but not always) better, it also takes more CPU power and time to compress and decompress, so it is typically used when space and bandwidth are a bigger constraint than time and CPU power.\\

To gzip or bzip a file, simply execute the respective command on the target file.  Note, however, that it will not create a new file. It will compress the target file and replace it, as well as adding a new file extension, either .gz or .bz2.\\

\begin{verbatim}
$ gzip destination.tar
$ bzip2 music.tar
$ ls 
destination.tar.gz    music.tar.bz2
\end{verbatim}

\subsection{Others}

While a combination of tar and either gzip and bzip2 is by far the most common archival and compression method used in the Unix world, there is still support for many other archive utilities which are usually available through software repositories or BSD ports. Among them are zip, xz, 7zip,rar, cpio, compress and many other lesser known utilities.

\section{The Mighty grep}

\textbf{grep}, as mentioned a few times earlier in the book, is a tremendously useful system administration tool.  It provides two primary functions:  filtering text streams and locating files containing certain text patterns. grep is so common and prevalent among Unix admins, that the program has become a verb, such as in the sentence, ”I had to grep through a few thousand log files.”

grep is used in one of two ways: in a Unix command pipeline or as aprogram in and of itself.

\begin{verbatim}
[jdoolin@thompson  ̃]$cat/var/log/messages | grep sda
[jdoolin@thompson  ̃]$ grep "randint" *.py
\end{verbatim}

The previous two commands do not display what the normal output would be. The first command would send the contents of the system logto the grep command, which would filter out any lines that contain thestring ”sda”. This could be a command used to search the system log forany information or errors on the hard disks in the system.  The second command will search all files ending in .py for the string ”randint”. This is an effective way to search the contents of files for certain text patterns.grep has a very useful flag, -v, that allows one to search for all lines that do not contain a particular string or pattern.

\begin{verbatim}[jdoolin@thompson  ̃]$ df -h | grep -v tmpfs
System                                Size   Used Avail Use% Mounted on
/dev/dm-0                              64G    46G   15G  76% /
udev                                   10M      0   10M   0% 
/dev/dev/sda1                         236M    33M  191M  15% /boot
/dev/mapper/thompson--home-home       266G    32G  221G  13% /home
\end{verbatim}

The previous command filtered out all lines that contain the string’tmpfs’, so that one does not see the many virtual filesystems in the ’df’output.

\section{dd - The Ultimate Data Copying Tool}

\textbf{dd} is among the most versatile and useful Unix commands at your disposal. Think of it like ’cp’ on steroids. Where ’cp’ can copy only whole files and directories, dd can copy only certain sections of files if you wish, or copy data into a certain area of a file. Better yet, dd can even access and use /dev/ devices, meaning you can directly access the data of devices such as hard disks, flash media and optical disks. This makes dd a useful tool for many tasks. Here are a few examples.

\subsection{dd: as an image or clone tool}

dd can be used to create images of CDROMs, DVDs and BluRays, floppydisks, USB drives, SD cards and even entire hard drives.

\section{head and tail}

\section{Logs}

\section{Cron}

\section{Backups}

\section{Single User Mode}

\section{SSH}

\subsection{screen}
