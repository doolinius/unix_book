\chapter{Unix Filesystems and Hard Disks}

Dealing with hard disks, optical drives and other storage media are one of the most common tasks with computer and systems administration.  This chapter covers many of the common utilities used to add, configure, manage and get information from hard disks and storage media on Unix systems.

\section{UNIX Directory Structure}

While not universal, most Unix systems, such as Linux, have a very similar directory structure.  Some of the directories found in typical Unix systems are analogous to some found on Windows systems, but not all of them.\\

One thing to keep in mind is that Unix treats nearly everything as files, even access to hardware devices as we shall see.

\subsection{/}

/ is the "root" directory of a Unix filesystem.  All disk drives and partitions will be accessible within this directory.  This is different from typical Windows setups where each partition or drive is designated its own drive letter.  All Unix systems, at the very least, \textit{must} have one partition on which / is mounted.\\

If you list the contents of /, you will see the top level directories of the system, which I will now discuss in more detail.

\subsection{/home and /root}

/home is the traditional directory that contains all user account directories or "home" directories.  This is analogous to C:\textbackslash Users on Windows systems.  Each user account with a valid login should have a home directory within /home, such as /home/jdoolin.  On MacOSX systems it is /Users.\\

/root is root's home directory.  But why wouldn't it simply be /home/root?\\

It is not uncommon in Unix network settings for /home to be mounted on a network drive, such as one provided by NFS or Samba.  This allows one to login on any Unix machine on a network, but still have access to the exact same content in their home directory.  It is also common for /home to be mounted on a different disk or partition.\\

If the /home mountpoint is unavailable, perhaps due to a network problem or a failing hard disk, then the users that have data stored within /home will then be unable to retrieve it.  The system is now potentially in a state that requires repair that will necessitate root access.  If root cannot access his own home directory, this could cause significant problems.  It is therefore a best practice to keep root's home directory on the \textbf{same partition as the root partition}, thus /root is rarely on its own partition or disk.\\

\subsection{/etc}

The bulk of Unix configuration files are located in /etc.  Think of it as analogous to the Windows registry, but usually consists of text files.  Key files are /etc/passwd, /etc/group, /etc/shadow, /etc/init.d and many, many others.

\subsection{/bin and /sbin}

Traditionally, /bin contains the primary system binaries required for basic system operation.  Such programs could be ls, mkdir, rm, rmdir, cp, mv, ps, kill, bash, chmod, cat and pwd.\\

/sbin typically contains system binaries that require root access.  These may include fdisk, fsck (filesystem check), mount and umount, ifconfig, route, or mkfs programs.\\

Some modern Linux systems do not use /bin or /sbin, and instead place all binaries in /usr/bin and /usr/sbin or even just /usr/bin.

\subsection{/usr}

The /usr directory is typically the largest aside from home directories.  It contains the majority of the rest of a Unix system's applications and all resources they require.  On Unix networks, it is common for /usr partitions to be mounted as a network share so that all users have access to the same applications that can now be managed on one server.

\subsubsection{/usr/bin, /usr/sbin and /usr/local/bin}

/usr/bin contains the majority of application programs, such as web browsers, compilers, text editors, audio and video applications and others that are needed by typical computer users.  On Arch Linux, however, /usr/bin contains nearly all binaries for the whole system.\\

/usr/sbin usually contains system services such as sshd, web servers, mail servers, directory servers, ntpd, nfs, samba, dns and dhcp servers and others.  It also contains some utilities often needed by users, such as tcpdump or traceroute.\\

/usr/local/bin often contains binaries local to the system or installed by means other than official package managers.  /usr/local will be discussed in more detail shortly.

\subsubsection{/usr/share}

Within /usr/share you may find platform independent resources required by other programs.  These aren't binary executables, but are files needed by other binaries.  This could be icons and images, audio files, documentation and html, time zone information, man pages and a dictionary of common words.  These files are in formats common to all operating systems and distributions.

\subsubsection{/usr/local}

Because some Unix networks mount /usr on a network share, it may be necessary for a local program to be installed that the network administrators do not wish to have on the file server.  /usr/local is a place for applications to be installed on the local machine.  It contains a set of directories that looks very similar to what is found in the / directory.  /usr/local has its own bin, share, sbin, include, etc and other directories.

\subsection{/var}

/var contains primarily variable data, or any data or files that change frequently, such as log files, print jobs, database files, caches and lockfiles.

\subsubsection{/var/log}

/var/log is a nearly ubiquitous directory for storing operating system log files, such as the main system log, login failures and successes, server logs (mail, web, ftp), firewall logs, boot logs or errors.

\subsection{/dev}

/dev is a unique and interesting directory.  It doesn't contain conventional files, such as text, image, audio, video or program data.  /dev contains files that refer to devices.  Some of these devices are physical, such as hard drives, the audio interface, keyboards, mice, joysticks, hard drive partitions, thumb drives, optical drives or terminals.  Other /dev entries are virtual, such as a file containing all zero data, random data and the mysterious /dev/null.\\

/dev entries can be read from and sometimes written to.  For example, the following command will read the contents of a CD or DVD ROM and send the output, bit for bit, to an ISO file:

\begin{verbatim}
$ dd if=/dev/sr0  of=/home/jdoolin/arch_linux.iso
\end{verbatim}

This command will read a floppy disk image and write it to a disk in the floppy drive:

\begin{verbatim}
$ dd if=boot_floppy.img of=/dev/fd0
\end{verbatim}

\subsection{/tmp}

As the name may imply, /tmp is a temporary storage location.  It is used by the operating system and is often also the only other directory on a system that is writable by normal users.  /tmp is usually emptied after every reboot, so do not use it as permanent storage.

\subsection{/boot}

On many Unix systems, /boot contains bootloader and bootloader configuration data, as well as the operating system kernel.  It is vital to the boot process.

\subsection{/lib, /usr/lib, /lib64}

Those familiar with Windows systems may be aware of .DLL files.  These are Dynamic Linked Libraries that provide functionality for a variety of other programs.  For example, a programmer doesn't need to write their own code for accessing the network because there will already be a library (.DLL) file for it.\\

In Unix, /lib and /usr/lib contain most of the libraries used by the operating system and applications.  On Linux, these files are typically .so files.

\subsection{Other common directories}

\subsubsection{/opt}

/opt is another directory that is self describing.  It is an optional directory that can be used when no other directory is quite appropriate.  This is usually in the form of locally installed, self contained application directories, such as those needed by google chrome, eclipse, android SDKs or other programs that extract to their own directory.  /usr/local can also be used for this purpose.

\subsubsection{/proc}

Like the /dev directory, /proc doesn't contain actual data files.  It is a virtual filesystem containing virtual files that have information about running processes and the kernel.

\subsubsection{/mnt and /media}

/mnt and /media are normally used for accessing removeable media or disks that are separate from the main operating system.  This inludes thumb drives, optical drives or hard drives with other operating systems installed.  Mountpoints may be permanently created, such as in the case of other hard drives or partitions within the system, but can also be created and deleted automatically, such as with USB, SD or other flash media.  The new directory will appear in /mnt when the drive is mounted, but then will disappear when the drive is removed.

\section{Hard Disk Related Commands}

This section specifically covers some useful utilities related to disks on Unix systems.

\subsection{df}

\textbf{df} (for "disk free") is a command that will show the current disk usage of \textit{all mounted filesystems}.  Note that it will not report usage for any partition or disk that is not mounted.  Example output:

\begin{verbatim}
$ df
Filesystem     1K-blocks     Used Available Use% Mounted on
/dev/sda1       76765216 18338260  54504372  26% /
/dev/sda3       40958972 28670548  12288424  70% /home
$
\end{verbatim}

The default output uses 1 kilobyte block counts, which may not be immediately readable by many users.  The "-h" command option enables "human readable" output as such:

\begin{verbatim}
$ df -h
Filesystem      Size  Used Avail Use% Mounted on
/dev/sda1        74G   18G   52G  26% /
/dev/sda3        40G   28G   12G  70% /home
$
\end{verbatim}

The command lists the filesystem block device as listed in /dev, the total size of the partition, how much is used, how much is available, percentage used and the mountpoint on which the partition can be accessed.\\

On many Linux systems you may also see entries for any dev, run and tmpfs filesystems also currently in use.  These do not refer to physical disks, but virtual filesystems.

\subsection{du}

To find how much disk space is used by a file or directory, the \textbf{du} command ("disk usage") command may be used.  Like df, du also has the "-h" option available.  Another common option is "-c" to show the total of all files and directories listed in the command.  Here is an example:

\begin{verbatim}
$ du -h unix_book
4.0K    unix_book/exroot/one/another
8.0K    unix_book/exroot/one
4.0K    unix_book/exroot/programs
77M     unix_book/exroot
78M     unix_book
$ du -hc unix_book
4.0K    unix_book/exroot/one/another
8.0K    unix_book/exroot/one
4.0K    unix_book/exroot/programs
77M     unix_book/exroot
78M     unix_book
78M     total
\end{verbatim}

This command shows the disk space used by the unix book directory.  This is a useful command for finding which directories and files are using the most space on a filesystem.  For very long file and directory listings, consider redirecting the output to a file.  Also consider using grep if you are searching for specific file or directory names within the output.

\subsection{fdisk}

While there are several disk partitioning options (such as the graphical utility \textbf{gparted}), fdisk is the ubiquitous program found on most Unix systems.  Note that fdisk usage requires root privileges.  fdisk may be used to list available disks and partitions with the "-l" option, but more importantly, may be used to edit the partition table of a disk, including adding, removing or changing partition information.\\

fdisk may be launched by simply supplying it a hard disk node in /dev, such as /dev/sda.  Once fdisk is launched it uses a menu driven system that prompts the user for information. Example:

\begin{verbatim}
$ sudo fdisk /dev/sdb
[sudo] password for jdoolin:

Welcome to fdisk (util-linux 2.24.1).
Changes will remain in memory only, until you decide to write them.
Be careful before using the write command.


Command (m for help):

Help:

  DOS (MBR)
     a   toggle a bootable flag
     b   edit nested BSD disklabel
     c   toggle the dos compatibility flag

  Generic
     d   delete a partition
     l   list known partition types
     n   add a new partition
     p   print the partition table
     t   change a partition type
     v   verify the partition table

  Misc
     m   print this menu
     u   change display/entry units
     x   extra functionality (experts only)

  Save & Exit
     w   write table to disk and exit
     q   quit without saving changes

  Create a new label
     g   create a new empty GPT partition table
     G   create a new empty SGI (IRIX) partition table
     o   create a new empty DOS partition table
     s   create a new empty Sun partition table
\end{verbatim}

From here the user may choose various options and will be prompted for further information depending on the option chosen.  The most common tasks will be deleting and adding partitions, printing the current partition table, viewing partition types and either quiting without saving changes or writing the table to disk and exit.  There will be a more detailed example later.

\subsection{mkfs}

When a new partition has been created it must then be formatted with a filesystem.  In the Windows world, the common filesystems are NTFS and FAT32.  A filesystem is the way the operating system organizes the data on a hard drive.  It also enables other features such as permissions, encryption and compression, journals that enable better recovery and other features.\\

In Linux systems, the command for creating a new filesystem typically begins with \textbf{mkfs} followed by a dot (.) and a filesystem name.  Examples would be mkfs.ext4, mkfs.ext2, mkfs.reiserfs, mkfs.xfs, mkfs.ntfs and mkfs.vfat.  Other systems, such as FreeBSD may have their own unique command such as 'newfs'.  Always read the documentation specific to your operating system or distribution.\\

The mkfs command requires at least one argument, and that is the partition itself, as it is listed in /dev.  For example:

\begin{verbatim}
$ sudo mkfs.ext2 /dev/sdb1
\end{verbatim}

This will create a new ext2 partition on the /dev/sdb1 partition.

\subsection{mount and umount}

Even if a partition has been formatted, it still cannot be accessed by applications until it is \textbf{mounted}.  In other words, the partition becomes attached to a directory in the Unix / file hierarchy.  That directory could be / itself, /home, /usr, /var, /tmp, /mnt/windows, /media/THUMB-DRIVE or even something like /home/jdoolin/media/music.\\

The mount command requires a partition as it is listed in /dev, a mountpoint directory and often a partition type specified by the "-t" option.  There may also be a number of options that can be passed for a filesystem, such as read-only support.  Here are a few examples:

\begin{verbatim}
$ sudo mount /dev/sda1 /
$ sudo mount /dev/sda2 /home
$ sudo mount -t ntfs /dev/sdb1 /mnt/windows -o remount,ro,noatime
\end{verbatim}

An exception for the mount arguments is found when a mountpoint is listed in /etc/fstab.  For example, if there is an entry in /etc/fstab to mount /home on /dev/sda2 and another to mount the cdrom drive (/dev/sr0) on /media/cdrom, then they could be mounted without specifying the partition.  Only the mountpoint is required:

\begin{verbatim}
$ sudo mount /home
$ sudo mount /media/cdrom
\end{verbatim}

It may also be necessary to unmount partitions.  When doing so, the command is \textbf{umount} (that is NOT a typo, it is umount and not UNmount) and the only option necessary is the mountpoint for the partition you wish to unmount.

\begin{verbatim}
$ sudo umount /media/cdrom
\end{verbatim}

\subsection{fsck}

\textbf{fsck} (for FileSystem ChecK) is analogous to the chkdsk command in Windows.  It is used to check for and repair errors in the filesystem.  fsck is often forced after a certain number of days or reboots.  It is also a good idea to use fsck when any filesystem or hard disk errors become apparent.  To check a filesystem simply type a command such as this:

\begin{verbatim}
$ sudo fsck /dev/sda2
$ sudo fsck /home
\end{verbatim}

Note that you can pass either a device name or a mountpoint to fsck.  

\section{Device Nodes}

\subsection{/dev Devices and Partitions}

On most Unix systems, hard drives and partitions will appear in the /dev directory as "block devices", which basically means any storage device that accesses its data in chunks of bytes or bits called "blocks".  The naming conventions for block devices varies between operating systems.  In Solaris, for example, a hard disk may be listed with a name similar to /dev/dsk/c0t2d0s3 (meaning controller 0, target 2, logical unit 0 and slice/partition 3).  FreeBSD disks may be named similarly to /dev/ad0s1 (disk 0, slice/partition 1).  In Linux, hard disks typically have the format /dev/sda, where the 'a' is the disk id.  If you have two disks, you may see /dev/sda and /dev/sdb.   Thumb drive inserted would add /dev/sdc.  Each partition is then listed as a number appended to the disk id.  So the first partition on /dev/sda would be /dev/sda1.  The second partition on the second disk would be /dev/sdb2 and so-on.\\

SD cards may appear as /dev/mmc0blk1.

\section{Various Filesystems}

While in modern Windows systems you are limited to using NTFS and FAT32 for hard disks, in the Linux and Unix world your choices are much more broad.  For the average user, the choice of filesystem doesn't mean much or warrant much thought.  But for those managing high performance systems or have specific needs, there may be filesystems available that are more appropriate for the task at hand.  The following are some of the filesystems currently available for use in Linux.

\subsection{ext2, ext3 and ext4}

\textbf{ext2, ext3 and ext4} have been the closest to a default filesystem in Linux.  ext2 was the first filesystem native to Linux and it was used as the default in many Linux distributions.  ext3 added many features to ext2, but of most importance is journaling, which is the ability of the filesystem to track its reads and writes, thereby minimizing damage during improper shutdowns or other possible filesystem problems.  ext4 is basically ext3 with a few more improvements.\\

All three are good general purpose filesystems though ext2 is still preferred for flash media such as solid state drives or USB storage.  Journaling is sacrificed for speed and fewer writes.  For most users, ext4 will be just fine.\\

The ext2,3,4 systems experience very low levels of fragmentation, so the need for a defragmentation utility is next to nil.

\subsection{btrfs}

\textbf{btrfs} is widely considered to be the next standard filesystem for Linux, particularly for high performance systems.  It is well suited for server and supercomputing environments and supports more advanced features such as resource pooling, snapshots of files and compression.

\subsection{XFS}

\textbf{XFS} was developed by Silicon Graphics (SGI) for use in their Irix Unix system.  It was developed with one thing in mind: very large files.  SGI workstations were used for developing and rendering 3D graphics, games and movies and thus required very large file support.  Being able to read and write large files as quickly as possible was of vital importance, so XFS supports this very well.\\

If you find yourself in a similar situation, where large media files will be a big part of your system and work, then XFS would be a good choice.  Note that you will sacrifice some hard drive space for the filesystem structures themselves, but you will gain performance.

\subsection{Reiser4}

\textbf{Reiser4} is more adequate for filesystems containing numerous small files such as logs, man pages, docs, icons, etc.  This makes it more suitable for directories such as /usr or /var.

\subsection{JFS}

\textbf{JFS} is a project by IBM tarteting their AIX Unix system and eventually the OS/2 operating system, but was ported to Linux.  It is a journaling filesystem, meaning it survives unintentional shutdowns very well.  Many claim that it uses less CPU resources and that it is an adequate system for numerous small files.  This may also be an adequate system for interoperability between Linux, AIX and/or OS/2.

\subsection{ZFS}

\textbf{ZFS} was created by Sun Microsystems (which was purchased by Oracle) as a highly advanced, scalable and reliable filesystem.  Its license is not compatible with the GNU Public License that the Linux kernel falls under, so it cannot be distributed with the Linux kernel.  It can, however, be installed as a third party kernel module (driver) and be used in Linux.  However, FreeBSD natively supports a free implementation of ZFS.\\

ZFS protects against data corruption, very high storage capacity, is scalable (meaning ZFS filesystems can be easily adjusted dynamically without having to create and re-create ZFS partitions), supports Access Control Lists for security, filesystem level compression and disk volumes without an additional layer, such as LVM.

\subsection{NTFS and FAT32}

NTFS and FAT32 are familiar to those in the Windows world.  It is possible read and write to these filesystems in Linux, which means you can use live CDs and bootable Linux USBs as data rescue systems.  You can also use Linux to create these filesystems for use on Windows systems.

\section{How To Add a Hard Drive}

\subsection{Install the Hard Disk}

First add the physical disk to the system and verify through the BIOS that the drive is working and recognized by the system.

\subsection{Find the Device Node}

When Linux boots it should find the new disk and create a new entry or entries in /dev.  If you have only one disk, /dev/sda, the next will likely be /dev/sdb, then /dev/sdc and so on.  If this disk has partitions already, they will also be present in /dev.  You can use the \textbf{dmesg} command or simply "ls /dev/sd*" to see the new entry in /dev.

\subsection{Partition the Disk}

Now repartition the disk if necessary.  Use fdisk on the disk entry in /dev:

\begin{verbatim}
$ sudo fdisk /dev/sdb
\end{verbatim}

Delete any existing partitions and create the new ones you wish to use using fdisk's menu system.  When finished, write the partition table to disk and exit.

\subsection{Format the Partition(s)}

You must now use an mkfs command to format any of the partitions you created on the new disk.

\begin{verbatim}
$ sudo mkfs.ext2 /dev/sdb1
\end{verbatim}

\subsection{Mount the Partitions}

Finally, you can now mount the formatted filesystem on the mountpoint of your choice:

\begin{verbatim}
$ sudo mount /dev/sdb1 /home/jdoolin/media
\end{verbatim}

\subsection{/etc/fstab}

If you wish this mountpoint and filesystem access to be persistent across reboots, you need to add an entry to fstab.  The format is mostly the same between Unix variants.  However, while some Unices still use device nodes such as /dev/ad0s3, Linux requires the UUID of the partition.  First find the UUID of the filesystem using the \textbf{blkid} command as root:

\begin{verbatim}
$ sudo blkid /dev/sda1
[sudo] password for jdoolin:
/dev/sda1: UUID="51dfcc2e-b71c-4357-b3e7-2f8878f5f19c" TYPE="ext2" PARTUUID="000d0f28-01"
\end{verbatim}

You can then use this UUID to add a line to /etc/fstab using your favorite text editor.  Here is an example from my Linux workstation:

\begin{verbatim}
UUID=216370fe-d3fd-425d-9e73-82aed041dce0 	/home		ext4 		defaults			0 2
\end{verbatim}

The information is separated by whitespace (tabs and spaces).  The first field is the UUID of the partition.  The second is the mountpoint where the partition will be accessed.  The third is the filesystem type (ext4, vfat, btrfs, xfs, etc).  The fourth is any arguments being passed to the mount program.  The fifth field is whether or not the filesystem information should be dumped and the final field is the order in which the disk should be checked.  The root system, /, should be 1 and all others 2.
