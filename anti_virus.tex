\chapter{UNIX Security}

\section{Linux Anti-Virus and Malware Removal}

Malware has a long and interesting history.  It was first simply annoyances, written by rather intelligent programmers seeking to show the world how clever they were.  It grew to be expensive, network hogging worms and system damaging viruses.  Eventually it became apparent that malware could actually be profitable, so rather than seeing a decrease in malware, there has actually been an \textit{increase}.

Malware comes in many forms, such as viruses, worms, trojans, spyware, adware, rootkits, browser hijackers, ransomware and many others.  It has become a concern for the safety of our private information, sensitive data and even our money.  With this in mind, we should be very mindful of preventing malware .

With this in mind, we need to take care in protecting our Linux and UNIX systems from the perils of malware.

\subsection{Just Kidding}

Hahahahahahahahaha.... malware on UNIX! Oh, that was a good one.  Hope you got as much of a laugh out of that one as I did.

One of the greatest advantages that Linux and UNIX users have is that malware is simply not a problem we have to deal with.  Oh, I already know what some of you are saying.  You're saying, ``Yeah, but Linux has such a tiny market share on the desktop, so why would malware writers even consider it as a target?  If Linux were more popular, it would have just as much malware as Windows''.

That seems like a legitimate argument at first, but when you consider the huge number of servers, routers, appliances and other network devices running Linux or some other UNIX variant, why would you be happy taking out a few thousand home PCs when you could take out a whole continent?

Or perhaps you've read about some Linux or macOS malware in the news and have concluded it's just as much of a problem as it is in Windows. Yes, it's true that every so often, some piece of malware is discovered that targets macOS or Linux. However, it often requires that the system have a particular Linux distribution with particular services running or configurations. It also usually requires the interaction of less informed users. But these are few and far between. 

There are a number of reasons why malware is more or less non-existant for Linux and UNIX.  Among them are:

\begin{itemize}
\item \textbf{Not running as root/administrator}.  Most UNIX users do not use the root account for day-to-day activities.  As such, malware is severely limited to what it could do on a UNIX system.  In order to infect binaries, the malware has to have root access, which is more difficult to achieve with accounts that do not themselves have root access.
\item \textbf{Software repositories}.  In Windows, when you install software, you either go buy it or you download it and install it from any number of sites.  A third option is to pirate the proprietary software.  Windows software can come from unreliable sources, especially if it's pirated.  When you install pirated software, there could be anything hiding in those executables.  Even when you install legitimate software, you often have to use the ``Custom Settings'' for ``Advanced Users'' to disable the installation of various toolbars and other malware.  On most UNIX systems, software is installed either from repositories or from packages provided on the original developers' sites.  This prevents malicious code from sneaking in.  Packages are also digitally signed to verify their integrity. It doesn't mean it's impossible, just far more difficult to do on a large scale. 
\item \textbf{Open source software}.  A lot of Windows software is proprietary or closed source, meaning that only the developers ever get to see the code.  Who knows what could be hiding in that program you just installed?  Open source software is an open book.  There are a lot of eyes on the source code, so malware has no chance to get in.
\item \textbf{Linux Distribution and UNIX variety}.  It would be difficult to write malware that works on all Linux systems, because not all distributions are built the same, use the same software or have the same configuration systems. This is further complicated by the variety in UNIX systems themselves, as malware for Linux is unlikely to work on FreeBSD, Illumos, or OpenBSD.
\item \textbf{User knowledge and experience}. - The typical Linux or UNIX user has a technology education and knows well enough to not click on suspicious email attachments, visit suspicious websites or otherwise fall for the social engineering tactics used to facilitate the spread of malware.  That's not to say it's true for all UNIX users, but it's a safe bet.
\end{itemize}

If you're running Linux, you can be pretty confident without running any anti-malware or anti-virus software.  I myself have used Linux, BSD and macOS systems for 18 years as of this writing and have never once had anything you could call a malware infection.

\section{But Seriously}

Having said all that, this doesn't mean you can just do whatever you want with a Linux system and you'll be fine for all time.  Just because malware isn't a problem doesn't mean security isn't a problem at all. It \textit{is} a problem.  A very big problem. 

Linux and UNIX systems still have security vulnerabilities outside of malware and thus they are still major targets for security attacks.  Webservers can be compromised, UNIX hosted web sites can be hacked, remote exploits can be conducted.  So you must remain vigilant with software updates and make sure you aren't running unneccessary services.

You should approach the security of Linux and UNIX system as seriously as you would any Windows system, even if malware isn't a concern.

\section{UNIX Security Practices}

Specific security measures may depend on your particular UNIX system, as Linux, FreeBSD, OpenBSD, macOS, Illumos or other UNIX systems may all have different available security measures. We'll touch on a few of these later.  However, there are some general policies that one can follow to mitigate security risk on UNIX systems.

\subsection{Keep Software Updated}

The majority of UNIX security breaches occur through vulnerable server software or web applications.  Still others occur due to root access being gained by unprivileged user accounts through vulnerable programs.  These can be mitigated by simply keeping your software up to date and applying patches, just as it does on a Windows machine.

For the truly security conscious sysadmin, it is advisable to keep aware of security vulnerabilities that are discovered for the systems you are managing and the software they use. FreeBSD, for example, will publish known discovered vulnerabilities on a mailing list, and even individual projects like Apache and MySQL will do the same.  Knowing what you are running can help you know what is serious enough to update. 

\subsection{Use Repositories and Package Managers}

By installing software from repositories using package managers, such as apt, yum, and pkg, you have a much lower risk of installing programs that contain any malware or malicious code.  These packages have been patched, tested and reviewed.  While it is still not impossible for there to be vulnerabilities, the likelihood of intentionally malicious code is very small. 

\subsection{Disable Unnecessary Services}

By turning off all unnecessary services you decrease the number of \textbf{attack surfaces} on the system.  Many UNIX systems enable certain services by default during installation.  If they aren't needed they should be disabled.  This can include mail servers, print servers, Samba file sharing or nameservice and others. The Irix system by Silicon Graphics was notorious for being insecure, but this was largely due to the large number of insecure services enabled by default. When these were disabled and more secure alternatives installed, Irix could be secured.

Most BSD systems enable only the bare minimum of services and require the admin to enable those they wish to run. The same is not true for Linux distributions, so always be mindful of what services are running after installing and disable those that are not needed. 

\subsection{Use Secure Remote Access}

Rather than use telnet or FTP protocols for remote access, use the OpenSSH alternatives.  While this isn't as much of an issue any more, it is still important for sysadmins to know that FTP and telnet are insecure, unencrypted clear text protocols. Usernames, passwords, files, commands, command output and all sorts of dangerous information can be read very easily by anyone sniffing packets on a replicated port or a Man in the Middle attack.  OpenSSH mitigates this possibility by using asymmetric (public/private key) encryption. 

However even OpenSSH must be protected.  SSH servers have a variety of configuration options that can help reduce the chances of being compromised.  For example, it is a good idea to disable root SSH access.  This means that even someone who stole or guessed the root password would not be able to use SSH to login with the root account.  SSH can also be used to allow only certain users or groups to login remotely, or even to disallow password authentication entirely, requiring all users to use public/private SSH key authentication.

Taking it a step further would include only allowing SSH connections from inside the network, or even from particular IP addresses.

\subsection{Proper User and Group Management}

When creating users and granting permissions to various resources, use the Principle of Least Privilege.  In other words, when creating users, only grant them access to the minimal resources required to perform their duties.  For example, you may not want to give them sudo access or put them in the BSD wheel group.  Perhaps you don't want them access to certain folders, drives or devices.

It is also a good practice to enable user account expiration.  This allows accounts to be automatically disabled after a certain amount of time or lack of activity.  Orphaned or abandoned accounts are potential security problems.

\subsection{Use a Firewall}

Most UNIX systems have firewalls either built-in or as optional software.  Linux has \textbf{iptables}, which is being replaced by \textbf{nftables}.  The BSD systems have \textbf{pf}, \textbf{ipfw} and \textbf{ipfilter}.  Enable and configure the firewall to block all network connection attempts other than those from legitimate sources and to legitimate services.

The typical firewall strategy is to block everything incoming by default, then ''poke holes'' in the firewall to allow the traffic you need through.  Firewalls can be very specific as well, such as allowing Port 22 traffic (SSH) but only for certain IP addresses.  

Firewalls should be configured at the network level and host level.  Network firewalls are systems through which all traffic must pass and the firewall filters it as needed, blocking all traffic that is not allowed.  There are numerous network firewalls available from many vendors, but it is also possible to build one with a UNIX or Linux system.  \textbf{pfSense} is a very popular FreeBSD based firewall distribution, while IPFire and ClearOS are Linux based firewall distros. One need not even use a particular distribution, and can build one from a base OS install.

\subsection{Make Regular Backups}

The three key aspects of Information Security are Confidentiality, Integrity and Availability. There is a huge focus on Confidentiality, but many often overlook the importance of Integrity and Availability.  Integrity means the data is accurate and unmodified from its intended state.  Availability means data is not lost or inaccessible.  

Keeping good backups helps to ensure Availability and Integrity.  It prevents loss of data and gives administrators the freedom of reinstalling compromised operating systems and restoring any data that may have been modified. Even using good storage systems such as FreeBSD's ZFS promotes data integrity even by protecting against hardware faults.  Using RAID, ZFS or Linux software raid ensures that data is not lost due to failed disks.

\subsection{Encryption}

For sensitive data, it is a good idea to use encryption. This can take many forms, from single file encryption to archive encryption, encrypted containers and even entire disk encryption.

PGP/GPG can be used to encrypt and decrypt files, as well as verify their senders. This is often done in email clients, but can also be performed on the command line.

Encrypted containers can be created by VeraCrypt (also available on Windows and macOS). These containers are something like folders that can even be moved around and transferred to other systems.

Whole disk encryption can also be done with VeraCrypt but also by the Linux utility dm-crypt and in FreeBSD with GELI, which can be enabled at install. However, be mindful that while whole disk encryption is beneficial, it can also result in misery when one must attempt to recover data from an encrypted disk that is not in its original PC. This depends on the encryption method used, but always read documentation and keep good backups.

\subsection{Intrusion Detection and Rootkit Detection}

There is an open source IDS (Intrusion Detection System) available called tripwire. This is a signature based IDS that is used to scan for and detect unauthorized changes to key operating system files. Rootkits often require replacing low level operating system files with compromised versions that hide themselves. Tripwire will detect when any of these files are changed or an attempt is made to change them.

It is also advisable to regularly run rootkit scans using tools such as rkhunter.

\subsection{Use Virtualization}

Particularly useful for servers, virtualization allows one to run many instances of Linux virtual machines, each running its own service. For example, it is possible on a LAMP machine to run each service on its own VM. This means that if only one VM is compromised, such as Apache, the MySQL server will remain protected.

Depending on the CPU features, Linux supports \textbf{Kernel Virtual Machines} (or KVM), meaning that the Linux operating system provides low-level access to hardware, increasing the speed and efficiency of virtual machines. FreeBSD does the same through it's \textbf{bhyve} hypervisor. 

FreeBSD has the unique \textbf{jails} feature, similar to Linux \textbf{Docker} containers, which are very similar to virtual machines. These are lightweight virtual machines that do not require full machine emulation.  It is sometimes referred to as \textbf{os-level virtualization}.  Linux and FreeBSD can run each service in its own isolated 'contaienr' or ’jail’. Each of these is a stand-alone copy of the operating system (minus the kernel) in separate folders. Each of these runs in its own memory space and cannot access any data outside of itself, meaning that if one jail were to be compromised it would leave the others safe.  Jails and containers can be easily cloned, backed up or deleted, making them very easy to manage.

\section{Summary} 

While malware may not be a major concern for UNIX systems, the usual security risks still apply. UNIX systems can still be hacked and compromised and are popular targets. Systems must still be secured through whatever means necessary, including software updates, good user management, firewalls, secure remote connection protocols and configurations, disabling services and many other techniques.

Good sysadmins need to be aware of the systems they manage and be aware of the vulnerabilities that may affect them, and take the steps to mitigate the risks.