\chapter{L.A.M.P}

\textbf{L}inux.\\

\textbf{A}pache.\\

\textbf{M}ySQL.\\

\textbf{P}HP.\\

This set of four open source projects is often referred to as the \textbf{L.A.M.P. Stack}.  It consists of a Linux distribution, the Apache web server, the MySQL relational database server and the PHP web programming language.  With these four pieces of software, one can set up a fully functioning web server capable of serving professional, data driven websites.  Many web hosting services rely on the LAMP stack for its services.\\

However, it doesn't necessarily need to be this exact combination of software.  It may be a FreeBSD system with the NGINX web server, PostGRES database server and Ruby on Rails web programming language.  Even with the different software, the concept is still the same: four open source projects with the goal of serving professional web sites.\\

If you are a web designer or are interested in web design, you owe it to yourself to become comfortable with this software.

\section{Web Server Software}

The web server is software that runs in the background, listens on a network interface and receives data from the operating system as it comes in through the network.  The web server receives requests for documents, HTML, CSS, images, flash players and any other data hosted on the server required to display the page.  There are many web server options, both proprietary and open source, but by far the most common are the open source web servers.  In particular, Apache.

\subsection{Apache}

Apache is practically the foundation of the majority of the World Wide Web.  It has had the greatest market share of web servers for many years.  It's an open source project, meaning its code is seen by millions, reviewed, patched quickly and enjoys excellent security\footnote{Keep in mind, however, that web servers may be vulnerable through the applications that it hosts.  Even a simple login page may be an entry into the whole system.}.  It is known for being very stable and capable of handling very large installations.  It supports virtual hosts, which means the ability to host many web sites on one system.  It is also extensible with its module system, allowing you to include only features that are needed.

\subsection{nginx}

nginx is a relative newcomer to the web server party, but it is a big one.  It was designed to be very high performance with low memory requirements, as well as being able to serve as a proxy for several network protocols and as a load balancer.  It has a simpler setup than Apache, but has a shorter history.

\subsection{lighttpd}

lighttpd is worth mentioning because it has a totally different philosophy: to be a strong, capable but very lightweight web server.  While it is convenient for smaller systems, it is also capable of running very large installations, such as Wikimedia.

\section{Database Server}

The database server is the process that allows a connection to a relational database system.  The relational database is how most website data is stored.  From small database systems such as SQLite and Access, to large scale databases that run on Oracle or Bigtable, the world runs on databases.  User accounts, product listings, social media posts, order history, recipes, student information, course offerings, billing systems and countless other systems have a database storing the information.\\

As with web servers, there are many options for relational database systems (RDBMS).  There are proprietary systems such as Microsoft SQL Server and Oracle and open source systems such as MySQL/MariaDB, PostGRESQL and SQlite.

\subsection{MySQL/MariaDB}

\subsection{PostGRES}

\subsection{SQLite}

\section{Web Programming Languages}

Once you have a web server for providing pages and documents and a database for storing all of the valuable data, you need some way of tying them together.  A way to retrieve data from the database and present it to the user.  A way for the user to change, add or remove data.  A way to present this data differently depending on what it is.  This is where the web programming language comes into play.\\

The web programming language allows the developer to write code that builds HTML documents on the fly and populate them with data retrieved from the relational database.  They are often built with elaborate frameworks that enable very quick development and connecting code with databases.\\

Nearly any language can be used as a programming language for a web site, but there are some that are far more common than others.  Among the common languages are ASP, Perl, PHP, Python, Java and Ruby (on Rails), but perhaps the most popular is PHP.\\

\subsection{PHP}

\subsection{Ruby on Rails}

\subsection{Python}

\section{A Basic Web Page}
